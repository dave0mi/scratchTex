
%%%%%%%%%%%%%%%%%%%%%%%%%%%%%%%%%%%%%%%%%%%%%%%%%
\documentclass[12pt,letterpaper,oneside]{article}

% Packages
%\usepackage{sectsty}
%\usepackage{caption}
%\usepackage{subcaption}
%\usepackage{pslatex}
\usepackage{times}
\usepackage{amssymb,amsfonts,amsmath,amscd}
\usepackage{accents}
%\usepackage[pdftex]{hyperref}
\usepackage[pdftex]{graphicx}
%\makeatletter
%\define@key{Gin}{resolution}{\pdfimageresolution=#1\relax}
%\makeatother

%\usepackage{todonotes}
%\usepackage{fnpos}
%\usepackage{caption}
%\usepackage{longtable}
\usepackage{verbatim}
%\usepackage{fancyhdr}
\usepackage{theorem}
\usepackage{float}

% Some handy commands
\newcommand{\norm}[1]{\left\Vert#1\right\Vert}
\newcommand{\abs}[1]{\left\vert#1\right\vert}
\newcommand{\set}[1]{\left\{#1\right\}}
\newcommand{\Real}{\mathbb R}
\newcommand{\Complex}{\mathbb C}
\newcommand{\eps}{\varepsilon}
\newcommand{\To}{\longrightarrow}
\newcommand{\Ker}{\textup{Ker}}
\newcommand{\Img}{\textup{Img}}
\newcommand{\diag}{\textup{diag}}
\newcommand{\circulant}{\textup{circ}}

\DeclareMathOperator*{\argmin}{argmin} 
\newcommand\munderbar[1]{ \underaccent{\bar}{#1}}

\newenvironment{prf}[1]
    {\indent\textit{Proof#1:}}
    {~\hfill $\blacksquare$\\}
    
% Fix some error reporting
\vfuzz2pt % Don't report over-full v-boxes if over-edge is small
\hfuzz2pt % Don't report over-full h-boxes if over-edge is small

% Set the paragraph ski
\setlength{\parskip}{3pt}

% ---------------------------------------------------------------
\begin{document}

\title{Curves}
\author{Dave Ilstrup}
\date{}
{\sf \maketitle}

\section{Some closest point solutions}

Given a curve $c(t):\mathbb{R}\rightarrow\mathbb{R}^n$
and a point $Q\in\mathbb{R}^n$, find $t^*$ s.t.
$|c(t)-Q|$ is minimized.
Basic approach is to solve $\frac{d}{dt}|c(t)-Q|=0$,
which gives Eqn\ref{eq:dotmin}. In general, these could be maxima
as well. In the examples here, we are interested in `finite' segments 
of parameterized curves ($t\in[0,1]$). So the results have to be checked
to avoid maxima and solutions outside $[c(0),c(1)]$.

\begin{equation}
\label{eq:dotmin}
<c,c'>=<Q,c'>
\end{equation}


A few examples: A line segment (Eqn\ref{eq:lineseg})
and a quadratic Bezier curve (Eqn\ref{eq:quadBezier}).
\begin{equation}
\label{eq:lineseg}
C_i(t) = c_i + t (c_{i+1}-c_i)
\end{equation}

\begin{equation}
\label{eq:quadBezier}
B_{p_0,p_1,p_2}(t) = p_1 + (1-t)^2(p_1-p_0) + t^2 (p_2-p_0)
\end{equation}


\begin{equation}
t^* = \argmin{t} |B_{p_0,p_1,p_2}(t)-Q|
\end{equation}


%%%%%%%%%%%%%%%%%%%%%%%%%%%%%%%%%%%%%%%%%%%%%%%
\subsection{Fun with Chebyshev polynomials}

$T_0 = 1$

$T_1 = x$

$T_{n+1} = 2xT_{n} - T_{n-1}$


\end{document}


